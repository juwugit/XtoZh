\chapter{CMS detector and LHC}
This thesis is done via analyzing the data collected by the Compact Muon Solenid (CMS) detector at the Large Hadron Collider (LHC). CMS is one of the two large general purpose detectors built on the LHC. The goal of CMS experiment is to investigate a wide range of physics, including the search for the Higgs boson, beyond standard model particles, extra dimensions, and dark matter. This chapter will briefly introduce the LHC and the CMS detector.


\section{Large Hadron Collider}
The LHC is the world's most powerful proton-proton collider and the largest experimental facility ever. It was built by the European Organization for Nuclear Research (CERN) between 1998 and 2008 in collaboration with over 10,000 scientists and engineers from over 100 countries, as well as hundreds of universities and laboratories. It lies in a tunnel 27 km in circumference, as deep as 175 m beneath the France$-$Switzerland bordr near Gevena. The designed maximum collison energy and highest luminosity of the LHC are 14 TeV and $\textup{10}^{-34} \textup{cm}^{-2}\textup{s}^{-1}$ respectively.\\
Other accelerators that had been originally built at CERN for previous experiments is working as an injection chain for the LHC now. The proton beam starts from LINAC, a small linear accelerator, where its energy firstly reaches 50 MeV. It then passes through a booster to the PS where it is accelerated up to 25 GeV, and then go to the SPS up to 450 GeV. The beam is finally injected in the LHC ring and is accelerated to 4 TeV by the case for 2012 datasets. In the fisrt half of 2015, it reached 6.5 TeV. This analysis is base on the data collected in the 8 TeV center-of-mass energy collision.\\



