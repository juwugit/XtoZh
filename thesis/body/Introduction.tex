\chapter{Introduction and theory overview}

\section{Introduction}
This thesis presents the analysis details and the results of the search for heavy resonances decaying into a Z boson and a Higgs boson (h) at the center-of-mass energy of 8 TeV, using 19.7 $fb^{-1}$ p-p collision data. In turn, the Z boson is identified through its leptonic decays (Leptons often refer to $e$ and $\mu$ only in experiments. $l = e, \mu$). The Higgs boson h is expected to hadronically decay into a pair of b-quarks. The investigated final states consist of two charged leptons which are identified in the detector and limit the presence of the background, and two b-quarks from the hadronic Higgs decay which collects the largest possible fraction of Higgs events.
\newline This thesis is written in the following order. In the latter part of this chapter, the heavy resonances model is introduced, including the expected cross section and the specification of model parameters. In chapter 2, the LHC and the CMS experiment are introduced, including the information of each sub-detector and the trigger system of the CMS. The details of the analysis are shown in chapter 3. This chapter reveals the way to reconstruct physical objects in CMS. By adding some proper kinematic selections on those physics objects, the interested event in data collected by the CMS detector can be selected. Moreover, this chapter shows the comparison between data and simulation. In the last chapter, the results of the search and the conclusion are shown.
