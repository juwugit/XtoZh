\chapter{Introduction and theory overview}

\section{Introduction}
This thesis presents the process and result of the search for signal of heavy resonances decaying into a Z boson and a Higgs boson (h) final state at center-of-mass energy of 8 TeV, using 19.7 $fb^{-1}$ p-p collision data. In turn, the Z boson is identified through its leptonic decays (Leptons often refer to $e$ and $\mu$ only in experiments. $l = e, \mu$). The Higgs boson h is expected to hadronically decay into a pair of b-quarks. The investigated final states consists of two charged leptons which are identified in the detector and limit the presence of the background, and two b-quarks from the hadronic Higgs decay which collects the largest possible fraction of Higgs events.
\newline This thesis is written in the following order. In latter of this chapter, the heavy resonances model is introduced, including the expected cross section and model parameter specification. In chapter 2, The LHC and the CMS experiment are introduced, also about the information of each sub-detectors and the trigger system of the CMS. The analyzing process is shown in chapter 3. This chapter reveals the way to reconstruct physical objects in CMS. By adding some proper kinematic selections on those physics objects, the interested event in data collected by the CMS detector can be picked up. Moreover, this chapter shows the comparison between data and simulation, there are various kinematic distributions including the reconstructed mass of the heavy resonances. In the last chapter, the final and conclusion of the search are shown.

