% This textfile includes
% 3. Analysis procedures
%    ..
%    ..
%    ..
%    3.3 Event and Object selection
%    3.4 Data and MC comparison
%    3.5 Background estimation
%

\section{Event and Object selection}


\subsection{Lepton Requirements}

\subsection*{Muon Selection}

Besides the muon ID criteria disscussed in section 3.3.1 (table~\ref{tab:MuonIDtable}), we also require kinematic cuts on the muon candidates. We select the transverse momentum of the leading muon candidate must greater than 40 GeV, while the second leading muon transverse momentum threshold is 20 GeV. All muon candidates are must in the psuedo-rapidity $|\eta| < 2.4$ region.

\subsection*{Electron Selection}
Kinematic cuts on the electron candidates are also neccesary. Although the electron ID selection (table~\ref{tab:EleIDtable}) already required the pseudo-rapidity of electron supercluster, we cut on the $|\eta| < 2.5$ for electron candidates and all electrons must out of [1.4442,1.566] in the $\eta$ region. The $p_{T}$ requirement is a bit different from the muon case. Since the HLT trigger already selects electron $p_{T}$ greater than 33 GeV, we require both leading and sub-leading electrons $p_{T}$ greater than 40 GeV in advance.

\subsection*{Jet Selection}
