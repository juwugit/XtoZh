% This chapter includes
% 3. Analysis procedures
%    3.1 Data sets and MC samples
%        3.1.X sub-sections for samples
%    3.2 Trigger
%    3.3 Physics objects reconstruction and identification
%        3.3.X sub-sections for physics objects

\chapter{Analysis Procedures}

In this chapter, the analysis procedures of the search for $Z'$ decaying into $Z$h in $llbb$ final state are reported. The data sets and Monte Carlo (MC) samples we used in this analysis will be indicated. Physics objects reconstruction and event selections are also introduced. Moreover, background yields and the effects of systematic uncertainties will be demonstrated in the end of this chapter.

\section{Monte Carlo Samples and Data sets}

\subsection{Signal MC}
As introduced in section 1.2.3, the signal hypothesis is HVT model B benchmark. The heavy resonance ($Z'$) is tested using a wide set of masses from 800 GeV to 2000 GeV, one masspoint every 100 GeV (Table~\ref{tab:TableSignalMC}). The signal is simulated by MadGraph5\_aMC$@$NLO\cite{MG5} in LO mode, as a narrow spin-1 neutral resonance and is forced to decay in the $Z'\rightarrow Zh\rightarrow llqq$ channel. Showering and hadronization are performed with PYTHIA6\cite{PYTHIA}.
\begin{figure}[hbtp]
  \begin{center}
    \includegraphics[width=0.4\textwidth]{figure/CH3/ZPrimeTo2l2q.png}
  \end{center}
  \caption{\label{fig:ZPrime2l2q}Feynman diagram for $Z'\rightarrow Zh \rightarrow 2l2q.$}
\end{figure}
\begin{center}
  \begin{table}
    \begin{center}
      \begin{tabular}{|c|c|c|}
        \hline
        Sample & Number of Processed Events & $\sigma_{LO}$(pb) \\ \hline
        ZPrime\_ZH\_lljj\_M800-MADGRAPH & 10710 & 0.00685367 \\ \hline
        ZPrime\_ZH\_lljj\_M900-MADGRAPH & 10209 & 0.00485861 \\ \hline
        ZPrime\_ZH\_lljj\_M1000-MADGRAPH & 19997 & 0.003263 \\ \hline
        ZPrime\_ZH\_lljj\_M1100-MADGRAPH & 9370 & 0.00217483 \\ \hline
        ZPrime\_ZH\_lljj\_M1200-MADGRAPH & 10710 & 0.00145484 \\ \hline
        ZPrime\_ZH\_lljj\_M1300-MADGRAPH & 9369 & 0.000979745 \\ \hline
        ZPrime\_ZH\_lljj\_M1400-MADGRAPH & 10497 & 0.000664783 \\ \hline
        ZPrime\_ZH\_lljj\_M1500-MADGRAPH & 19999 & 0.000454339 \\ \hline
        ZPrime\_ZH\_lljj\_M1600-MADGRAPH & 8950 & 0.000312541 \\ \hline
        ZPrime\_ZH\_lljj\_M1700-MADGRAPH & 9369 & 0.000216282 \\ \hline
        ZPrime\_ZH\_lljj\_M1800-MADGRAPH & 10708 & 0.000150398 \\ \hline
        ZPrime\_ZH\_lljj\_M1900-MADGRAPH & 10498 & 0.000105039 \\ \hline
        ZPrime\_ZH\_lljj\_M2000-MADGRAPH & 19999 & 7.36377e-05 \\
        \hline
      \end{tabular}
    \end{center}
    \caption{\label{tab:TableSignalMC}Signal samples used in the analysis.}    
  \end{table}
\end{center}
\newpage
\subsection{Background MC}
Since we are looking for new resonances decaying in semi-leptonic final state, the background samples of this analysis are originated by all SM events with two leptons and at least one jet as final state. The dominant background contribution is the produciton of Z boson with jets. This Z+jets sample is produced by MADGRAPH. In the matrix element level, the Z boson is forced decaying into two leptons, and further this sample is divided into two samples depending on the Z $p_{T}$, higher than 100 GeV or between 70 and 100 GeV. The contribution of events with Z $p_{T}$ less than 70 GeV is negligible due to further cut on the objects $p_{T}$ in the selection criteria.

The second dominant source of background is $t\bar{t}$ production. Both of the two top quarks decay into all leptonic final state (top decays into a W boson and a b quark first) which gives two leptons, neutrinos and two b-jets. This sample is generated by POWHEG\cite{POWHEG}.

Other sources of background considered are SM di-boson productions (WW, WZ and ZZ) generated by PYTHIA6. All the background samples are required to pass phase-space cuts, $p_{T}^{ll} > $60 GeV and 60$ < M_{ll} < $120 GeV. Related statistics are reported in Table~\ref{tab:TabBkgMC}.

\begin{center}
  \begin{table}[h]
    \begin{center}
      \begin{tabular}{|c|c|c|}
        \hline
        Sample & Number of Processed Events & $\sigma_{NLO}$(pb) \\ \hline
        DYJetsToLL\_PtZ-70To100 & 11764538 & 63.5 \\ \hline
        DYJetsToLL\_PtZ-100 & 12511326 & 39.4 \\ \hline
        TTTo2L2Nu2B & 10783509 & 25.8 \\ \hline
        WW & 7759752 & 56.0 $\pm$ 2.3 ($\pm$ 0.3) \\ \hline
        WZ & 9910267 & 22.4 \\ \hline
        ZZ & 9769891 & 7.6 $\pm$ 0.3 ($\pm$ 0.3)\\
        \hline
      \end{tabular}
    \end{center}
    \caption{\label{tab:TabBkgMC}Background samples used in the analysis.}    
  \end{table}
\end{center}

\subsection{Data Samples}
In this analysis, the full CMS data collected in 2012 is used, corresponding to the integrated luminosity of 19.7 fb$^{-1}$ at center-of-mass energy $\sqrt{s} = $8 TeV. For each lepton channel, there are four datasets. All datasets are collected with a double muon or a double electron trigger, as explained in detail in the next section. The trigger algorithm employed for the electron samples doesn't use any information from the tracker but only the energy deposite in the ECAL. This expedient is implemented in order to avoid any possible inefficiencies due to the presence of two tracks very close to each other when the Z is highly boosted and its decay products are very collimated. Such a trigger is contained in the Photon/DoublePhotonHighPt dataset. The full dataset names are listed in Table~\ref{tab:TabDataSet}.

\begin{center}
  \begin{table}[h]
    \begin{center}
      \begin{tabular}{|c|c|}
        \hline
        AOD Sample & Luminosity (pb$^{-1}$) \\ \hline
        DoubleMu/Run2012A-22Jan2013-v1 & 876.225 \\ \hline
        DoubleMuParked/Run2012B-22Jan2013-v1 & 4409 \\ \hline
        DoubleMuParked/Run2012C-22Jan2013-v1 & 7017 \\ \hline
        DoubleMuParked/Run2012D-22Jan2013-v1 & 7369 \\ \hline
        Photon/Run2012A-22Jan2013-v1 &  876.225 \\ \hline
        DoublePhotonHighPt/Run2012B-22Jan2013-v1 & 4412 \\ \hline
        DoublePhotonHighPt/Run2012C-22Jan2013-v1 & 7055 \\ \hline
        DoublePhotonHighPt/Run2012D-22Jan2013-v1 & 7369 \\
        \hline
      \end{tabular}
    \end{center}
    \caption{\label{tab:TabDataSet}Data sets used in this analysis.}    
  \end{table}
\end{center}

\section{Trigger}
Since the final state contains two same flavour leptons and at least one jet, we perform this analysis on the DoubleMu and Photon/DoublePhotonHighPt datasets. The first dataset is triggered by two muons, the second one is triggered by two eletrons. These triggers are:
\begin{itemize}
\item HLT\_Mu22\_TkMu8* (for DoubleMu datasets)
\item HLT\_DoubleEle33\_* (for Photon/DoublePhontonHighPt datasets)
\end{itemize}

The muon trigger has a double $p_{T}$ threshold, requires leading muon $p_{T}$ greater than 22 GeV and sub-leading muon $p_{T}$ greater than 8 GeV. Differently, the double electron trigger requires a higher threshold of 33 GeV to electrons. The trigger efficiencies are close to 1 in both cases.

\section{Physics Objects}

\subsection{Muon}
\subsection*{Reconstruction}
The muon reconstruction algorithm at CMS takes advantage of the redundancy of detection methods. Muon tracks are first reconstructed independently in the inner tracker (tracker track) and in the muon system (standalone track). Based on these objects, two reconstruction approaches are used\cite{MuonReco}:
\begin{itemize}
\item $Globol~Muon$ (outside-in): Starting from a standalone track, this algorithm finds a best tracker track to match the standalone track. Then, the fit of the track is repeated using the hits both in the tracker and in the muon system\cite{KF}. The resulting object is called a $Global~Muon$. At large transverse momentum ($p_{t} > $200 GeV), the global muon fit can improve the momentum resolution compared to the tracker only fit.
\item $Tracker~Muon$ (inside-out): A tracker muon is reconstructed by an opposite direction from a global muon. In this approach, all tracker tracks with $p_{T} >$ 0.5 GeV and the total momentum $p >$ 2.5 GeV are considered as possible muon candidates. The extrapolation to the muon system takes into account the magnetic field, average expected energy losses, and multiple scattering in the detector material. If at least one muon segment matches the extrapolated track, the corresponding track track qualifies as a $Tracker~Muon$. This algorithm is useful for low-$p_{T}$ muons that are not fully penetrate the muon system, and therefore only register a few hits
\end{itemize}

If no match is found when extrapolating outside-in, the standalone track is stored as a $Stanalone~Muon$. This happens only for less than 1\% of the muons produced in a collison, and the reconstruction efficiency is about 99\% for the muon which carries enough high momentum within detector coverage\cite{MuonReco}.

\newpage
\subsection*{Identification}
We use both tracker muons and global muons in this analysis. To identify muons from the signal, the muons must pass one of these two off-line selections, high$-p_{T}$ muon ID or tracker-based muon ID\cite{MuonID}. The requirements are listed as follows:\\

High-$p_{T}$ muon ID
\begin{itemize}
\item Muon identified as a $Global~Muon$.
\item Number of muon hits in the global track $> 0$.
\item Number of matched muon stations $> 1$.
\item Number of pixel hits $> 0$.
\item Number of tracker layer with hits $> 8$.
\item Transverse impact parameter $d_{xy} < 0.2$ cm.
\item Longitudinal impact parameter $d_{z} < 0.5$ cm.
\item Relative error on the track transverse momentum $\sigma_{p_{T}}/p_{T} < 0.3$.\\
\end{itemize}

In the tracker-based muon ID, the muon has to be identified as a $Tracker~Muon$, and the requirement of muon hits in the global track is removed. Other requirements are the same.

An additional useful variable for lepton identification is the isolation. It is defined as the scalar sum of the $p_{T}$ of the reconstructed objects within a cone (typical size is $\Delta R=0.3$) space around the lepton track but excluding the $p_{T}$ of the lepton itself. Moreover, the relative isolation is defined as isolation divided by the lepton $p_{T}$ ($I_{rel} = Iso/p_{T}^{lep}$). The relative isolation is more frequently used in the modern analysis.

In this analysis, a modified isolation criteria is used. The two muons originated from boosted Z decay are close to each other, and consequently the presence of another muon in the isolation cone could break the function of this variable. To solve this problem, we exclude the energy contribution of muons inside the cone.

\begin{align}
  \label{eq:ModIso}
  I_{rel}^{mod}=\frac{\sum p_{T}^{CH}+max(0.0 , \sum E_{T}^{NH}+\sum E_{T}^{\gamma}-0.5\times \sum p_{T}^{PU})}{p_{T}^{lep}}
\end{align}

In the above equation, $p_{T}^{CH}$ denotes the charged hadron transverse momentum inside the cone. Similarly, $E_{T}^{NH}$ and $E_{T}^{\gamma}$ stand for neutral hadron and photon transverse energy respectively. The last term in the nomimator, $\sum p_{T}^{PU}$ is defined as sum of transverse momentum of the charged particles in the cone but with particles not originating from the primary vertex (for pile-up corrections). Finally, the modified requirement is $I_{rel}^{mod} < 0.1$.\\

\begin{table}[h]
  \begin{center}
    \begin{tabular}{|l|c|c|}
      \hline
      Variable & Standard & Modified \\ \hline
      Muon type & Global muon & Tracker muon \\
      Muon hits in global track & $\geq$ 1 & - \\
      Muon stations matched & $\geq$ 2 & $\geq$ 2 \\
      $d_{xy}$ & $<$ 0.2 cm & $<$ 0.2 cm\\
      $d_{z}$ & $<$ 0.5 cm & $<$ 0.5 cm\\
      Pixel hits & $\geq$ 1 & $\geq$ 1 \\
      Tracker layers & $\geq$ 8 & $\geq$ 8 \\
      $\sigma_{p_{T}}/p_{T}$ & $<$ 0.3 & $<$ 0.3 \\ \hline
      $I_{rel}^{mod}$  & $<$ 0.1 & $<$ 0.1 \\
      \hline
    \end{tabular}
    \caption{\label{tab:MuonIDtable} Summary of the muon ID selection criteria.}
  \end{center}
\end{table}


\subsection{Electron}
\subsection*{Reconstruction}
Text.
